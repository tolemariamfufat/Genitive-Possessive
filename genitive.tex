\documentclass[11pt,a4paper]{article}
\usepackage{tipa}
\usepackage{apacite,pslatex}
\usepackage{titling}
\title {Genitive Construction in Afaan Oromoo}
\author {Tolemariam Fufa}
\date{September 2022}
\begin{document}
	\newcommand\keywords[1]{%
		\begingroup
		\let\and\\
		\par
		\noindent\textbf{Keywords:}\\#1\par
		\endgroup
	}
	\maketitle
	\begin{abstract}
		In this paper, I describe Afaan Oromoo Genitive Construction. In Afaan Oromoo, genitive constructions are marked by lengthening a final vowel of a noun. 
		A noun that got its vowel long is interpreted as the possessor a noun that precedes it. The interpretation of Afaan Oromoo genitive construction where 
		two nouns are involved is straightforward; but, if more than two nouns are involved ambiguities arise. This means that if a genitive construction is 
		formed by three nouns, the second noun can be interpreted either as possessor (concerning the preceding noun) or as the possessed noun (concerning the 
		following noun). I hypothesize that a genitive construction that involves more than two nouns is a borrowed instance that lacks an established phrasal 
		pattern. These ambiguities are the source of public arguments. In the first part of this paper, I shall discuss problem areas of the topic by citing 
		examples from social media. Moreover, I shall discuss more complex examples of Afaan Oromoo genitive constructions by citing sample MA thesis titles. 
		In the second part, I shall discuss literature reviews on Afaan Oromoo genitive construction. The third part proposes solutions and concludes the paper. 
		
	
		
		
	\end{abstract}
	\keywords{}
	\newpage
	\section{Introduction}
	
	Genitive is a grammatical category that relates two or more than two nouns and often known as a case marking indicating possession or close relationship or attachment. Genitive is classified in different ways by different people for different purposes. Some literature classify genitive into five types as possessive, subjective, source, objective and descriptive \cite{ELT,vassiliadis1985translation}. Some other reseachers classify genitive construction into two main categories; namely, possessive genitive and descriptive genitive constructions; possessive genitives being further classified into two main categories as alienable  (ALP) and inalienable possession (IAP). \cite{gebregziabher2012alienable,ELT}. 
	
	In this paper, I describe Afaan Oromoo Genitive Construction. I classify Afaan Oromoo genitives into two types: possessive and descriptive genitives. Source, object, subject and other subcategories of genitive construction shall be categorized under descriptive genitive while ALP and IAP genitives shall be discussed under possessive genitives.
	
	In Afaan Oromoo, genitive constructions marked in five ways: (1) A noun ending in long vowel marked by null morpheme. For example, in ’mana Jiraataa’ the dependent (possessor) ’Jiraataa’ has long final vowel; it doesn’t show any change to indicate genitive construction. (2) If a noun ends with short vowel, we mark genitive case by lengthening the short vowel \cite[p,183]{gragg1976oromo}. For example, in mnana namaa ’a man’s house’, the short vowel of the nouns nama ’man’ becomes long to indicate genitive construction. (3) If a noun ends in consonant, we mark the genitive case by suffixation of -ii \cite{gobena2019verb}. For example in ’warqii ilkaanii’ the noun ilkaan ’teeth’ ends with n, to which -ii is suffixed. A noun that got its vowel long is interpreted as the possessor a noun that precedes it. (4) The genitive can be made by placing the pronoun kan ’whome’ between the head and the dependant (possessor) nouns; for example mana kan k’ottuu ’a farmers’s house’ (Gragg, 1976:183). In Meca dialect the pronoun kan is used for both masculine and feminine genders. But in Harar, Borena and Tullama dialects kan shows masculine while tan is used for feminine gender. For example, mana tan ishee ’her house’. (5) Genitves of genitives which are not final genitives are in absolute form \cite{owens1985grammar}. \\	
	
	\indent 1a. mana namaa ’a man’s house’\\
	\indent 1b. mana nama Sudaan ’a sudanese house’ \\
	
	The interpretation of Afaan Oromoo genitive construction is clear if two nouns are involved; but, if more than two nouns are involved ambiguities arise. This means that if a genitive construction is formed by three nouns, the second noun can be interpreted either as possessor (concerning the preceding noun) or as the possessed noun (concerning the following noun). In this paper I shall address problem areas of genitive construction by citing examples. Moreover, I shall discuss more complex structures of Afaan Oromoo genitive constructions. Finally, I shall also propose solutions and conclude the paper. 
	
	
	
\section{Possessive Genitive}
	
 
	 Possessive genitives are further classified into two main categories; these are alienable possession (ALP) and inalienable possession (IAP). ALP and IAP are mainly differentiated on the bases of their corresponding semantic relationships. IAP forms a fixed semantic relationship between the possessor and the possessee (Gebregziabher, 2012: 161). For example, the genitive construction usually associated to owenership of something. But, close investigation shows that genitive construction includes possession, source, origin and description. In Afaan Oromoo the genitive case is is marked in five ways: (1) By zero morpheme (2) By lengthening the short final vowel of the possessor; (3) By adding long vowel -ii to a possessor ending in consonant; (4) By putting the pronoun kan/to between the head and the dependant (possessor) nouns; (5) By applying dissimilation to genitive of genitives; 
	
		The alienable-inalienable asymmetry: Evidence from Tigrinya. In Selected Proceedings of the 42nd Annual Conference on African Linguistics (pp. 161-182).
	
	The genitive case is often treated as possessive case. But all issues concerning genitive case are not explainable interms of possession. For example, 
	1a. aarii-n nam-icha-a kolfa dubartii sana-af qab-u guddaa-dha. rage-NOM man-DEF-POSS laughter woman.POSS that-to has-INF big-be ’The rage of the man towards laughter a woman is high.’ 
	(1a) informs us about rage and laughter of two participants. Eventhough, the sentence is a gentive case it fails to explain the issue interms of possession because it is not likely we consider a rage and a laughter as something one possesses. in aarii-n nam-icha-a ’the man’s rage’ the genitive is marked by lengtheing the final short vowel while in kolfa dubartii ’a woman’s laughter’ the gentive is marked by zero morpheme. 
	It is possible to express analytical aarii namicha and kolfa dubartii as in: 2a. Nam-ich-i aarii qab-a man-DEF-NOM rage has-3MSS ’The man has rage.’ 
	2b. Dubartii-n kolfa qab-d(t)i. woman-NOM laughter has-3FSS ’A woman has laughter.’ 
	The following table shows variations of genitive meaning in Afaan Oromoo: 
	Example Hiika Type of genitive 
	1. Mana namaa ’man’s house possessive 
	2. Ergaa firaa a relative’s message origin 
	3. Aarii nam-ichaa the mans’s rage subjective 
	4. Hidhamuu ishee her imprisonment objective 
	5. mana qorichaa medicine house descriptive 
	As shown above there are five different types of genitive. 
	Inalienable possessions and alienable possessions. 1a. gurra mucaa ear child ’a child’s ear’ 
	1b. abbaa gurbaa father boy ’a boy’s father’ 
	As different from IAP, ALP doesn’t form a fixed semantic relationship between the possessor and the possessee. The semantic relationship between the possessor and the possessee depend on the context (Gebregziabher, 2012: 161). For example, 
	1. mana barsiisaa house teacher ’a teacher’s house’ 
	As shown in (2), there is no fixed semantic relationship between ’house’ and ’teacher’ because ’a teacher’s house’ can be interpreted as a house that a teacher bought, built, rented and so on. Some languages like Tigrinya make a grammatical distinction between inalienable and alienable posssesions (Gebregziabher, 2012: 161); while others like Afaan Oromoo do not make a grammatical distiction between the two. 





	
	
	
	Gebregziabher, K. (2012). The alienable-inalienable asymmetry: Evidence from Tigrinya. 
	In Selected Proceedings of the 42nd Annual Conference on African Linguistics (pp. 161-182).
	
	Genitive construction can be classified into two main categories; namely, possessive genitive and descriptive
	genitive constructions. Possessive genitives are further classifed into two main categries; these are alienable possession (ALP)
	and inalienable possession (IAP). ALP and IAP are mainly differentiated on the bases 
	of their corresponding semantic relationships. IAP forms a fixed semantic relationship between the possessor and the possessee (Gebregziabher, 2012: 161). For example,
	
	\newpage
	\bibliographystyle{apacite}
	\bibliography{gen}
	
	\newpage
	\section*{Acronyms and symbols}
	
\end{document}