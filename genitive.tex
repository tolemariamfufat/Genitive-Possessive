\documentclass[11pt,a4paper]{article}
\usepackage{linguex}
\usepackage{tipa}
\usepackage{apacite,pslatex}
\usepackage{titling}
\title {Genitive Construction in Afaan Oromoo}
\author {Tolemariam Fufa}
\date{October 2022}
\begin{document}
	\newcommand\keywords[1]{%
		\begingroup
		\let\and\\
		\par
		\noindent\textbf{Keywords:}\\#1\par
		\endgroup
	}
	\maketitle
	\begin{abstract}
		In this paper, I describe Afaan Oromoo Genitive Construction. In Afaan Oromoo, genitive constructions are marked by lengthening a final vowel of a noun. 
		A noun that got its vowel long is interpreted as the possessor a noun that precedes it. The interpretation of Afaan Oromoo genitive construction where 
		two nouns are involved is straightforward; but, if more than two nouns are involved ambiguities arise. This means that if a genitive construction is 
		formed by three nouns, the second noun can be interpreted either as possessor (concerning the preceding noun) or as the possessed noun (concerning the 
		following noun). I hypothesize that a genitive construction that involves more than two nouns is a borrowed instance that lacks an established phrasal 
		pattern. These ambiguities are the source of public arguments. In the first part of this paper, I shall discuss problem areas of the topic by citing 
		examples from social media. Moreover, I shall discuss more complex examples of Afaan Oromoo genitive constructions by citing sample MA thesis titles. 
		In the second part, I shall discuss literature reviews on Afaan Oromoo genitive construction. The third part proposes solutions and concludes the paper. 
		This paper attempts to describe Afaan Oromoo genitive construction.	This paper is divided into 6 sections. The first section is an introduction to Afaan Oromoo genitive construction. Section 2 is discusses possessive genitives of Afaan Oromoo; the concern of this section is alienable possessive (ALP) and inalienable possessive (IAP) of Afaan Oromoo. Section three discusses descriptive genitive of the language. Section 4 deals with problem areas of genitive of genitives of Afaan Oromoo. Section 5 proposes solutions for problems of genitive of genitives. Section six concludes the paper.
	\end{abstract}
	\keywords{genitive, possessive, descriptive, alienable, inalienable}
	\newpage
	
	\section{Introduction}
	\label{sec:org462cb4a}
	
	Genitive construction is treated in different ways in literature. For teaching and translation purposes genitive construction is often typed into 5 categories. These are possessive, subjective, source, objective and descriptive genitive \cite{ELTcon}. Researchers such as Rosenbach categorize genitive construction into two categories; namely, possessive and descriptive genitive \cite{rosenbach2006descriptive,gebregziabher2012alienable}. 
	
	
	Subjective and objective genitives are used when a noun is derived from a verb and modified by another dependent noun. If the dependent noun expresses the main noun that is derived from the original verb, it is claimed to be a subjective genitive; but, if the dependent noun (genitive) expresses the object of the main verb, the genitive is said to be objective genitive. In subjective genitive an action or idea is set forth as proceeding from the noun in subject position \cite[68]{greenlee1950genitive}. The subjective genitive expresses about the nature of subject of a sentence. It does’t refer to possession. 
	
	
	In Afaan Oromoo subjective and objective genitives cannot distinguished morphologically or syntactically; they are expressed in the same with construction. Thus, subjective and objective ginitives are distinguished semantically. Therefore, context is the main determining factor in this case. \\
	
	\ex. 
	\ag.
	nam-ni isa jaal-at-a\\
	man(people)-NOM him love-MID-3MSS\\
	'People loves him.'\\
	
	\ex. 
	\ag.
	inni nama jaal-at-a\\
	he.NOM man(people) love-MID-3MSS\\
	'He loves people.'\\
	
	\ex. 
	\ag.
	jaalala nama-a\\
	love man(people)-GEN\\
	'People's love or love of people.'\\
	
	As shown above \emph{jaalala} 'love' is derived from the verb \emph{jaalata} 'loves'. The dependent is \emph{nama} 'people' in this context. The dependent \emph{nama} express the subjective genitive on the bases of example (1) provided that people loves him. If we take into consideration example (2), \emph{jaalala namaa} 'people's love' is also subjective genitive as long as \emph{inni} 'he' dedicated to welfare of humanity. But if we are talking about general observation love of human being for human being, \emph{jaalala namaa} 'love of people' becomes objective genitive. 
	
	
	Moreover, a source can be expressed by a genitive case \cite[69]{greenlee1950genitive}. \\
	
	\ex.
	\ag.
	caama bona-a\\
	drought summar-GEN\\
	'summer drought'\\
	
	\ex.
	\ag.
	lolaa ganna-a\\
	flood winter-GEN\\
	'winter flood'\\
	
	In (4) \emph{bona} 'summer' is considered to be the source of drought; and in (5) \emph{ganna} 'Winter' happens to be the source of flood. In both cases genitive is marked by lengthening the final short vowels of the corresponding nouns, \emph{bona} and \emph{ganna} respectively. 
	
	In order to clarify source genitive, let us consider the following examples:
	\ex.
	\ag.
	buna Kafaa\\
	coffee Wallaggaa.GEN\\
	'Kafaa coffee'\\
	
	\ex.
	\ag.
	k'amadii Arsii\\
	wheat Arsii.GEN\\
	'Arsii Wheat'\\
	
	\ex.
	\ag.
	hoolaa Wolayitaa\\
	sheep Harar.GEN\\
	'Wolayita sheep'\\
	
	Examples given in (6-8) are source genitives. (6) expresses coffee is from \emph{Kafaa}; the sources of wheat and bull are also expressed in the same way, namely from \emph{Arsii} and \emph{Wolayitaa} respectively. Because the dependent nouns in (6-8) have long final vowels, the genitive is marked by null morpheme. 
	
	Materials of which a thing is made can be expressed by the genitive case \cite[69]{greenlee1950genitive}.
	\ex.
	\ag.
	siree sibiilaa\\
	bed iron.GEN\\
	'iron bed'\\
	
	\ex.
	\ag.
	foon hoolaa\\
	meat sheep.GEN\\
	'mutton'\\
	
	\ex.
	\ag.
	mana \textipa{\!d}agaa\\
	house stone.GEN\\
	'stone house'\\
	
	In (9), the material from which a bed is made, in (10) a type of meat and in (11), a material from which a house is built are expressed by genitive \emph{sibiilaa} 'iron', \emph{hoolaa} 'sheep' and \emph{\textipa{\!d}agaa} 'stone' respectively.
	
	Not only materials of which a thing is made but also a purpose a material for which it is planned my be expressed by genitive. In this case word order is the reverse of examples given in (9-11).
	
	\ex.
	\ag.
	siibiila siree\\
	iron bed.GEN\\
	'bed iron'\\
	
	\ex.
	\ag.
	foon hoolaa\\
	meat sheep.GEN\\
	'meat sheep'\\
	
	\ex.
	\ag.
	\textipa{\!d}agaa manaa\\
	stone house.GEN\\
	'house stone'\\
	
	As shown above word order of genitive phrases given in (12-14) are the reverse of examples given in (9-11) to show that purpose and material genitives are expressed not only morphologically but also syntactically.
	
	A whole thing from which only a part is to be focused may expressed by genitive construction known as the partitive genitive \cite[:69]{greenlee1950genitive}. Although most partitive 
	constructions refer to a quantity or amount, some are used to indicate quality or behavior.  
	\ex.
	\ag.
	halkan walakkaa\\
	night half.GEN\\
	'middle of the night'\\
	
	\ex.
	\ag.
	lit'a gannaa \\
	beginning winter.GEN\\
	'beginning of winter'\\
	
	\ex.
	\ag.
	baha gannaa\\
	end winter.GEN\\
	'end of winter'\\
	
	\ex.
	\ag.
	barii Dilbataa\\
	dawn Sunday.GEN\\
	'dawn of the Sunday'\\
	
	The above given examples indicate time. In (15), \emph{halkan walakkaa} 'middle of the night' indicates not only amount of time (which is roughly 6 hours) but also quality or behavior linked to middle of the night which is silent and fearsome. Similarly, (16-18) indicate amount of time as well as behaviors associated with \emph{lit'a gannaa} 'beginning of winter', \emph{baha gannaa} 'beginning of winter' and \emph{barii Dilbataa} 'dawn of the Sunday'. In the beginning of winter farmers prepare themselves for farming season. By contrast end of winter is associated with new year celebrations and time of happiness and joy. Dawn of the Sunday is also linked with time of rest, worship and socialization. 
	
	
	In this paper I categorize genitive construction into two types: possessive and descriptive genitives. Descriptive genitives are genitive constructions which are not possessives \cite{rosenbach2006descriptive}.
	
	In Afaan Oromoo, genitive constructions marked in six ways:\\
	
	(1) A noun ending in long vowel marked by null morpheme. For example,  in \emph{mana Jiraataa} 'Jiraataa's house' the dependent (possessor) \emph{Jiraataa} has long final vowel; therefore, there is no overt genitive marking in such case. That is to say the possessor \emph{Jiraataa} doesn't show any morphological change to indicate genitive construction. \\
	
	(2) If a noun ends with a short vowel, we mark genitive case by lengthening the short vowel \cite[183]{gragg1976oromo,gobena2019verb}. For example, in \emph{mana namaa} 'a man's house', \emph{nama} 'man' has final short vowel in absolute form; the final short vowel becomes long in genitive construction. \\
	
	(3) If a noun ends in consonant, we mark the genitive construction by suffixation of \emph{-ii} \cite{gobena2019verb}. For example in \emph{wark'ii ilkaanii} 'golden teeth' the noun \emph{ilkaan} 'teeth' ends with \emph{n}, to which the genitive mark \emph{-ii} is suffixed. A noun that got its vowel long is interpreted as the possessor of a noun that precedes it. \\
	
	(4) The genitive can be made by placing the pronoun \emph{kan} 'whose' between the head and dependent (possessor) nouns; for example \emph{mana kan k'ottuu} 'a farmers's house' \cite[183]{gragg1976oromo}. In Meca dialect the pronoun \emph{kan} is used for both masculine and feminine genders. But in \emph{Harar, Borena and Tullama} dialects \emph{kan} shows masculine while \emph{tan} shows feminine gender. For example, \emph{mana tan ishee} 'her house'. \\
	
	(5) Genitves of genitives which are not final retain their absolute form \cite[104]{owens1985grammar}. \\
	
	1a. mana namaa 'a man's house'\\
	1b. mana nama Jimmaa 'a Jima man's house'\\
	
	(6) Genitive construction can also be indicated by possessive pronouns. For example,\\
	\\
	2a. mana koo 'my house'\\
	2b. mana kee 'your house (sing)'\\
	2c. mana ishee 'her house'\\
	2d. mana isaa 'his house'\\
	2e. mana keenya 'our house'\\
	2f. mana keessan 'your house (pl)'\\
	2g. mana isaanii 'their house'\\
	
	The interpretation of Afaan Oromoo genitive construction is clear if two nouns are involved; but, if more than two nouns are involved ambiguities arise. This means that if a genitive construction is 
	formed by three nouns, the second noun can be interpreted either as possessor (concerning the preceding noun) or as the possessed noun (concerning the
	following noun). A genitive construction that involves more than two nouns happens to be ambiguous and becomes source of public arguments. 
	
	
	
	\section{Possessive Genitives}
	
	Genitive case often employed to express possession (Greenlee,1950:68)
	Possessive genitive can be expressed either morphologically or syntactically. For example, 
	1a. mana Tolasaa
	1b. Tolasaan mana qaba
	2a. obboleessa Boontuu
	2b. Boontuun obboleessa qabdi.
	The above examples are possessive genitives. They are expressed morphologically and syntactically. 
	Possessive genitive is devided into two types: alienable and inalienable possessions.
	
	\subsection{Alienable Possessives (ALP)}
	
	Alienable possessions refers to possessions which have not fixed semantic relationships between the 
	possessor and the possessed nouns. That is to say alienable possessions can freely change owenership. These
	includes materials such as car, house, computer, book, etc. We can say,
	3a. Jabeessaan mana qaba
	3b. mana Jabeessaa
	4a. Koortuun kitaaba qabdi
	4b. kitaaba Koortuu, etc
	
	Alienable genitive constructions are expressed syntactically. These genitive constructions show possession. Possessive genitives are alienable genitives.
	
	\subsection{Inalienable genitive}
	
	Inlienable genitives are possessive cases which can be expressed morphologically. As compared to these genitives, an inalienable genitives cannot 
	expresseed morphologically. For example,
	
	3a. Gaaddiseen refeensa dheeraa qabdi.
	3b. *rifeensa dheeraa Gaaddisee
	4a. Waariyoon ilkaan kaarruu qaba
	4b. *ilkaan kaarruu Waariyoo
	
	Gebregziabher, K. (2012). The alienable-inalienable asymmetry: Evidence from Tigrinya. 
	In Selected Proceedings of the 42nd Annual Conference on African Linguistics (pp. 161-182).
	
	Genitive construction can be classified into two main categories; namely, possessive genitive and descriptive
	genitive constructions. Possessive genitives are further classified into two main categories; these are alienable possession (ALP)
	and inalienable possession (IAP). ALP and IAP are mainly differentiated on the bases 
	of their corresponding semantic relationships. IAP forms a fixed semantic relationship between the possessor and the 
	possessee (Gebregziabher, 2012: 161). For example,
	
	The genitive construction usually associated to owenership of something. But, close investigation shows that
	genitive construction includes possession, source, origin and description. In Afaan Oromoo the genitive case is 
	is marked in five ways: 
	(1) By zero morpheme 
	(2) By lengthening the short final vowel of the possessor;  
	(3) By adding long vowel -ii to a possessor ending in consonant;  
	(4) By putting the pronoun kan/to between the head and the dependant (possessor) nouns;   
	(5) By applying dissimilation to genitive of genitives;
	
	
	The genitive case is often treated as possessive case. But all issues concerning genitive case are not
	explainable interms of possession. For example,
	
	1a. aarii-n   nam-icha-a     kolfa     dubartii   sana-af qab-u   guddaa-dha.
	rage-NOM  man-DEF-POSS   laughter  woman.POSS that-to has-INF big-be
	'The rage of the man towards laughter a woman is high.'
	
	(1a) informs us about rage and laughter of two participants. Eventhough, the sentence is a gentive case it
	fails to explain the issue interms of possession because it is not likely we consider a rage and a laughter as 
	something one possesses. in aarii-n nam-icha-a 'the man's rage' the genitive is marked by lengtheing the final
	short vowel while in kolfa dubartii 'a woman's laughter' the gentive is marked by zero morpheme. 
	
	It is possible to express analytical aarii namicha and kolfa dubartii as in:
	2a. Nam-ich-i    aarii qab-a
	man-DEF-NOM  rage  has-3MSS
	'The man has rage.'
	
	2b. Dubartii-n kolfa     qab-d(t)i.
	woman-NOM  laughter  has-3FSS
	'A woman has laughter.'
	
	
	Inalienable possessions and alienable possessions.
	1a. gurra mucaa
	ear   child
	'a child's ear'
	
	1b. abbaa  gurbaa
	father boy
	'a boy's father'
	
	As different from IAP, ALP doesn't form a fixed semantic relationship between the possessor and the 
	possessee. The semantic relationship between the possessor and the possessee depend on the context (Gebregziabher, 2012: 161). 
	For example, 
	
	\begin{enumerate}
		\item mana    barsiisaa
		house   teacher
		'a teacher's house'
	\end{enumerate}
	
	As shown in (2), there is no fixed semantic relationship between 'house' and 'teacher' because 'a teacher's house'
	can be interpreted as a house that a teacher bought, built, rented and so on. Some languages like Tigrinya make 
	a grammatical distinction between inalienable and alienable posssesions (Gebregziabher, 2012: 161); while others like Afaan Oromoo
	do not make a grammatical distiction between the two. 
	
	\section{Descriptive Genitive}
	
	\begin{enumerate}
		\item how do descriptive genitives differ from possessive genitives?
		\item Are descriptive genitives syntactic, morphological or compounds?
		\item How do descriptive genitives differ from N + N sequences?
	\end{enumerate}
	
	Possessive genetives expands nominals into noun phrases. Semantically, possessive genitives specify (in)definiteness and establish
	reference within the NP. 
	In Afaan Oromoo the head can be separately determined by definite article or by other reference tracking devices:
	1a. -kitaaba Guyyoo
	1b. -Kitaabicha Guyyoo  (*the Johon's book)
	
	2a. -kitaaba namichaa 
	2b. -kitaaba namichaa kana (*this the man's book)
	
	Semantically, the possessor Guyyoo in (1a) functions like the definite article, specifying the referent of the NP. 
	In this example Guyyoo specifiees whose book it is, namely Guyyoo's. From a cognitive-pragmatic and semantic point of view
	the possessor can be viewed as an 'anchor' that narrows down the referent of the NP (Rosenbach, 2006:80). 
	
	In Afaan Oromoo the possessor can be postmodified and can be headed by a final determiner (note that English possessor can be pre- as well as 
	postmodified 
	and can be headed by an initial determiner):
	3a. kitaaba namicha guddaa [the big man]NP's book
	3b. kitaaba namicha kaleessa argitee [the man you saw yesterday]NP's book
	
	Genitive constructions in which the possessor functions as a determiner have NP status and they denote a specific
	entity. In 'kitaaba namichaa' the noun 'kitaaba' is a specific book. 
	
	In contrast the dependant in descriptive genitives is not an NP but usually a noun. 
	4a. mana dhagaa
	4b. *mana dhagichaa
	4c. manicha dhagaa
	In (4b) the definite article -icha- can only belong to the dependant 'dhagaa' and cannot belong the the head 'mana'. Semantically, (4b) is different 
	from (4a). 
	Therefore, the depandant is a nominal rather than a full NP in such cases can be seen from the ungrammaticality of 
	(4b). This shows that the dependent cannot have a determiner of its own. In (4c) -icha belongs to the head 'mana'. That is to say, the final determiner 
	goes with the head (and not
	the dependent). Therefore descriptive genitives are themeselves not full NPs but nouns or nominals and, in contrast to
	determine genitives, they denote properties and not specific entities. 
	
	Semantically, the dependent in descriptive genitives contributes to the denotation of the head noun, not specifying
	in (4a) whose house it is (as in a corresponding determiner genitive) but rather what type of house. As such, the 
	the dependant has a classifiying function in such genitives. As a classifier, the dependent is not referential and
	does not refer to a specific referent. Not that in 'mana dhagaa' reference is not made to specific stone 'dhagaa' 
	but to stone 'dhagaa' in general. 
	
	The semantic difference between determiner genitives and descriptive genitives as discussed above are reflected in 
	different positions in Afaan Oromoo noun phrases. Word order in the noun phrase is iconically determined in that any element contributing 
	to the denotation of the head noun is positioned adjacent to the head, while anything contributing to the reference
	of the noun phrase will be most distantly located away from the head noun (p,81). 
	
	5a. hoolaa foonii
	5b. *hoolaa foonichaa
	5c. hoolaa namicha sanaa 
	5d. hoolaa adii namicha sanaa
	(5a) shows word order of descriptive genitives. As shown in (5b) descriptive genitives prohibit the expansion, prefer
	the dependant to be adjacent to the head. (5c) shows determiner genitives which allows expansion of the noun phrase.
	In (5c) the determiner -icha and the 'sana' are added to the noun 'nama' which specifies sheep 'hoolaa'. Further, 
	the adjective 'adii' is inserted between the possessor and the head noun to expand the noun phrase to (5d) is determiner 
	genitive construction.(p,82).
	
	A descriptive genitive can be classifying, metaphorical, and generic one. It specifies another noun. 
	\begin{enumerate}
		\item Digrii Lammaffaa
		\item Gulantaa lammaffaa
		\item Itti aanaa Ministeeraa
	\end{enumerate}
	
	
	ARE DESCRIPTIVE GENITIVES SYNTACTIC PHRASES OR COMPOUNDS?
	
	(a) Coordination test
	
	Rosenbach, A. (2006: 83) gives three criterion to test if descriptive possessors are syntactic or cmpound. 
	First criterion is Coornation test. In Afaan Oromoo, coordination of dependant is common with descriptive genitives. 
	9a. foon hoolaa
	9b. foon reettii
	9c. foon hoolaafi reettii
	10a. mana dhagaa
	10b. mana mukaa
	10c. dhagaa manaa
	10d. dhagaafi muka manaa
	11a. reettii foonii
	11b. hoolaa foonii
	11c. reettiifi hoolaa foonii
	12a. hoolaa hormaataa
	12b. reetti hormaataa
	12c. reettiifi hoolaa hormaataa
	
	
	Because compounds do not allow a third element to be inserted between theme, these examples indicte taht descriptive genitives are syntactic phrases 
	and not compounds. 
	
	(b) Modification of the dependent
	
	If an N + N construction is a compound, then it should not be possible to separtely modify the first noun. 
	
	13a. foon hoolaa
	13b. foon hoolichaa
	
	
	In fact it should be noted that the dependent in descriptive genitives gives a different interpretation after modification.
	In hoolaa'sheep' is not a determiner, it expresses a type, not specifying foon 'meat'. 
	
	(c) Modification of the head
	
	As Rosenbach, A. (2006: 85) says, the strongest test for phrasehood is the ability of a modifier to intervene between
	the dependent and the head noun as shown below:
	14a. foon hoolaa
	14b. foonicha hoolaa
	15a. boojjitoo marqaa
	15b. boojjitoowwan marqaa
	
	As shown above -icha and -oowwan are added to the head by intervening between the dependent and the head noun
	to rule out compound status. 
	
	HOW DO DESCRIPTIVE GENITIVES DIFFER FROM N + N SEQUENCES? (p, 89) (there is no N + N) in Afaan Oromoo
	
	Descriptive genitives shown so far are known as classifying genitives. These genitives are said to be the prototypical cases 
	Rosenbach, A. (2006: 91). 
	
	\subsection{TYPES OF DESCRIPTIVE GENITIVES}
	
	Rosenbach, A. (2006: 92) argues there are three different functions of descriptive genitives: classifying, metaphorical and generic ones. 
	Classifying genitives are the ones usually referred to in the literature as 'descriptive genitives' (p, 92). These genitives
	are used to name certain objects and they can convey various degrees of lexicalization, from completely oqpaque expressions
	to fully semantically transparent ones (p, 92). 
	16a. ija bunaa
	16b. gumaa garbuu
	16c. arraba ibiddaa
	16d. guyyaa dubartootaa, Seera Makkoo Billii, 
	16e. harbuu Bantii
	16f. buqqee seexanaa (sheexanaa)
	(it includes, idiomatic expressions, plant names, insect names, named after person (p, 92). Product names, others
	\begin{verbatim}
		girl's school, writers block, spider's web ...etc (p, 93)
	\end{verbatim}
	
	
	In this naming function descriptive genitives are those that most clearly correspond to the term 'classifying genitives' as their basic
	function is type restriction. However, thhis only holds for semantically endocentric cases, where the dependent 
	clearly restricts the denotation of the head noun. In these cases the meaning of the head is the meaning of the 
	whole genitive NP, i.e. St Valentine's day designates a certain day, women's undrwea a certain type of underwear
	and smoker's cough a certain typeof cough, while a baby's head is not a type of head but a steak and kidney pudding.
	Similarly in (33d) the meaning of the whole genitive construction is not deducible compositionally from the meaning of the 
	head and the dependent attribute.. Rather, in these cases teh descriptive genitive refers to a complete mess (dog's breakfast) or to a specific type of balcony 
	(widow's walks). While they are not as such transparent, knowledge of the etymology of these idiomatic expressions makes them fully
	compositional in the figurative world, so to speak. For example, the term widow's walks (for porches on the roof)
	goes back to the fact (or rather legend?) that the wives of seafarers used to to climb up there to watch out for the 
	return of their husbands. Note, however, that even in the transparent cases (33e-g) the meaning of calassifying genives
	is much more restrictive than in a corresponding determiner genitive. Electrician's tape, for example, describes a specific
	type of tape, while a corresponding deterinmer genitive ([the elecrician's]tape) could mean various things: the 
	tape the electrician possesses, uses, wants to have, dreams of, or whatever. It is in the nature of possession to allow
	for all these meanings. As the function of the classifying genitives in (33) is to uniquely designate a specific object, it is 
	clear that not all these possessive  meanings carry over and taht , so to speak, one possessive meaning gets 'frozen' in these cases. 
	It is this 'freezing' of meaning which makes them so prone to undergo lexicalization and acquire lexeme  status. It is 
	presumably for this reason that Huddleston \& Pullum (2002: 470) regard this type fo genitive as 
	'a somewhat unproductive category'. They note, for example, taht while we can have a summer's day and a winter's day,
	a spring's day or an autumn's day are very questinable (they mark the latter two with a?). It is tru that the last two expressions are far less common than
	the first two. (p, 93). 
	
	Syntactic processes are by definition productive; it is usually only in the domain of word formation that the notion
	of productivity is evoked at all. As argued above descriptive genitives (in the sense of the calassifying genitive discussed here0
	can be syntactic phrases, and as such they should also be 'productive'. However, even when perceiving classifying genitives
	as being formed by the rules of word fromation, productivity is usually defined as the ability of a form to coin new
	expressions (p,94). That is, what matters is not how often a particular collection is used, but whether it is possible to coin
	it in the first place. In the examples above, it is the actual frequency of use that makes the difference. Thus, notice
	that both 'a spring's day and 'an autumn's day' can be used perfectly well to refer to a certain type of day; it is just
	that for seme reason they don't happen to be particularly frequent \ldots{} p, 94. While frequency of use may give us an indication
	as to the degree of lexicalization, it doesn't tell us anything about the productivity of the form/construction. 
	
	Note that, in general, new classifying genitives can be easily formed on the spot wheneve we perceive something as defining a type.
	Classifying genitives are fully productive (p, 94). 
	
	
	\subsubsection{Metaphorical genitives}
	
	Another type of descriptive genitives is used to describe an object, experience, state etc. in terms of another (p, 94). 
	one. 
	
	\begin{enumerate}
		\item akka dhala sooressaa (meeshaa mi'aawaa bituu jaalata)
		\item hamma baallee shimbirroo hin'ulfaatu
		\item 
	\end{enumerate}
	
	In (37a) the weight fo a swan's feather is used to describe the (light) weight of a person. In (37b) the well-known sound
	of a a dentist's drill characterizes the (ghastly) sound of a person's voice. The strength/intensity of the wind 
	is compared in (37c) to a prize fighter's blow. In (37d) the inner state of a person is described by means of the image
	of a child's inflatable toy. Note that in this last example the whole context is already in the metaphorical world:
	a word cannot lift a person literally. Rather, what is meant is that the word raises the person's spirits/mood, and 
	the way to get this image to the reader is to stay in this 'lifting' - metaphor and compare it with an inflatable toy
	(which can easily be lifted). The face in (37e) is compared to a child's pink ballon from which most of the air has escaped, 
	i.e. this image reinforces the description of the face as puffed and crumpled, and the colour of the ballon (pink)
	resembles the colour of the face. While in the examples in (37) the comparison is always made explicitly, most typically
	by expressions such as as, like , it reminded x of (p, 95). 
	
	
	In (38a) the person gives a yowl typical of cats. The person in (38b) does not of course literally have a puppy's 
	eyelashes, but eyelashes that look like those of a puppy. And in (38c) the child described is not a tennis umpire. 
	In fact, the whole setting is not a tennis setting, but it describes the settting of a play rehearsal by childeren,
	with the 'stage director' being physically located above her caste. That is, she does not literally have a tennis umpires's 
	advantage of height but only metaphorically -like a tennis umpire. (p, 95-96). 
	
	Note also that these metaphorical genitives can often be found in ellipitical constructions (p,96).
	
	In all these constructions the head of the genitive construction is omitted and refers back to a previously mentioned
	noun. This indicates taht such metaphorical genitives are phrases and not words. In many cases it is not clear whether the 
	initial article (usually the indefinet article) belongs to the dependent ofr to the whole genitive NP; however, examples (39a and c)
	show that (at least in these case) the singualr articcle goes with the (singular) dependent and not whth the (plural) head. 
	That is, structurally these genitives behave like determiner genitives. So, what -if anything - justifies their 
	classification as descriptive genitives?
	
	Note that in all the exampes avove the dependent is very clearly not specific. In fact, it cannot be, since the whole 
	possessive NP is not specific -as said, it serves as teh vehicle to transport a certain image into the (specific) context.
	Semantically, therefore, these construction are like descriptive genitives. They do not have the functin of a typical determiner, i.e. the referential
	anchoring of a referent, since the whole possessive NP as such is not specific. Rather, they evoke a certain typicl property 
	(hence, they are akin to what Strauss, 2004, calls 'propert-denoting possessives') (p, 96). 
	
	Under this view, we may regard these metaphorical uses of genitives as a type of determiner genitive operation in a 'subordinate mental space',
	facilitatin the identificaton of the referent of the (fictional) source domain. In this scenario we would first construe teh image
	of a fictioanl referent, i.e. a puppy, and then connect it to its eyelashes.. In this case we might argue that within this 
	fictional context, i.e. this subordinate mental space, a referent exists. That is, the dependent is specific in some irreal, 
	fictious world, but clearly unspecific in the  'real' world of discourse. Alternatively, however, we may also view them as a special case of 
	classifying genitive which is already akin to -but not identical with -determiner genitives. Under this view, then, 
	the unspecific dependent would not help to identify a (fictional) referent but a property. In this scenario, we construe
	eyelashes that are typical of puppies (in general), resulting in a nonsepecific interpretaion of the dependent. 
	Emperically, it is very difficult to decide which of the two conceptualization routes language users take (and see also
	section 4.1. on such constructional ambiguity). In any case, such metaphorical uses are very common. Not surprisngly, they are 
	most often found in fictional texts, such as novels, where metaphor is a common device to get across certain imaginative
	images or ideas. Metaphorical genitives are productive in the same way that determiner genitives ar, since they are modelled on them (p, 96-97)
	
	
	\subsubsection{Generic genitives}
	
	So far, we have only looked at the specificity of the dependent as a typical property distingushing determiner and 
	descriptive genitives. Under this view, the depedent in a descriptive genitive is unspecific in its naming function and 
	in tis comparing function, while it is specific in a determiner genitiv. However, specificity alone cannot account
	for all the referential properties genitive-marked dependents can have in English (Afaan Oromoo???). In this section I argue that it is important 
	to distinguis,, within the calass of nonspecific dependents, nonreferring dependents from referring (generic) dependents.
	
	Nonspecificity and genericity are often treated alike, but they are different notions with different properties. While specificity
	is a notion used to capture the referential properties of indefinite NPs, 'generic noun phrases are those in which 
	reference is made to an entire class' (C.Lyons, 1999:179). While specificity is a term usually only applied to indefinet
	NPs, generic NPs can also be expressed by definite NPs (The lion is a dangerous animal). Genericity is a concept that 
	can apply to both sentences and NPs. On the NP level, generic NPs have been considered as 'kind-referring' NPs by Krifka et al. (19995), 
	as opposed to 'object-referring' NPs, which denote a specific object/individual. A crucial difference between generic
	and nonspecific entities then is taht generics refer (to kinds), while nongeneric, nonspecific NPs don not refer at all.
	
	Now, how does all this relate to desriptive genitives? In (41) the dependent (a testator) is clearly not referring to 
	a specific testator. However, it does refer, as is evident from the susequent anaphor his which refers back to the 
	dependent(a testator). In contrast to determiner genitives this is not reference to a specific testator but referne to the 
	kind 'testaror', i.e. the dependent is 'kind-referring', i.e. generic.
	
	(41) Under family-provision legislation a testator's moral responsibility to provide for his dependents had become 
	a legal obligation in 1938 (p, 98).
	
	In contrast to metaphorical genitives, these genitives are not used to compare a referent in terms of another referent
	(or the referent's properties) but to describe a specific referent by setting it in relation to its kind. Note again
	that in the examples in (42) there is a specific refernt in the context that matches the generic possessor, a woman
	in (42a) 'a thin woman's apology', and a man in (42b) 'a man's horrible, humiliting sobbing'. In all these cases the 
	possessor abstracts away from the specific individuals, transcending them as examplars of a kind. Sometimes, this affinity
	of a specific referent in the context to its kind is even makde explicit, in a kind of mocking way, as in (43), or 
	or as describing the kind first and then finally identifying the matching (and co-present) referent, as in (44). 
	
	(43). So. What you're about, MrC.Shepherd. You here as the hopeful answer to a maiden's prayers? ..,
	(44) He was wanting a woman with a woman's way and a woman's knowledge, one who'd be as necessary to him as he was
	her. And you're that kind of woman, Val. \ldots{}(p, 98).
	
	There are other cases where the whole context is generic, and where therefore no potential specific referent is co-present
	and hence no ambiguity between the modes of speaking arises; example (45a) a woman's fear of breats cancer, (45b)
	a reader's respect, (45c) a woman's thought \ldots{} (p, 98).
	
	In all these examples the genitive NP receives a generic interpretation via the overall generic 'scripts' or 'frames'
	of the contexts \ldots{} That is, the overall context is generic and it is because of this overall generic setting that 
	the genitive therein is generic, too. However, withn this generic context the abstract refernts get individualized. 
	
	Note that we can also find generic dependents which are definite, while it was typical of the metaphorical genitives
	discussed in in secion 3.2. to have indefinete dependents. In fact, teh definite singual is often regarded as teh 
	prototypical generic expression \ldots{}
	
	In the examples in (46), 'the bad child's impulse to cry, the oversleepers's panicked sense of having fallen behind, 
	the optimistic Victorian's deep faith in progresss' \ldots{}the definite dependent is not specific but kind-referring;
	example (47) 'And he could do noting about his complexion, swarthier than the average Englishman's' illustrates that
	such generic definite dependents can also occur in ellipitic construction (p, 99). 
	
	Like metaphorical genitives, generic genitives behave structurally like determiner genitives, but semantically like
	descriptive genitives (with respect to their nonspecific dependent). Like metaphorical genitives, generic genitives are
	as productive as determiner genitives, as every individualized referent can be conceived of as a representaitve of its kind.
	(p,99). 
	
	
	As shown in the above examples (11) describes second degree, (12)  grade and (13) position. 
	
	\section{Problems Related to Genitive of Genitive in Afaan Oromoo}
	Biiroo
	Biiroo Misoomaa
	Biiroo Misooma Qabeenyaa
	Biiroo Misooma Qabeenya Bishaaniifi Inarjii Oromiyaa
	
	Biiroo Barnootaa
	Biiroo Barnoota Eegumsa Fayyaa
	Biiroo Barnootaa Eegumsa Fayyaa Oromiyaa
	
	The Flexibility of genitive in Afaan Oromoo
	Example                           Signaling
	koomee durba amuruu               'a physical attribute'
	obboleessa koo isa hangafa        'a blood relationship'
	abbaa manaa Uumee                'a non-blood relationship'
	alaqaa barreessituu              'a hierarchical relationship'
	hiriyaa barataa                  'a social relationship'
	garee Biiftuu                    'membership'
	beeksisa dhaabaa                 'performance'
	mana Jaalataa                    'ownership'
	gadda uummataa                   'emotional state'
	ergaa abbaashee                  'origin'
	kitaaba abbaashee isa jalqabaa   'human creation'
	dhimma biyyaa                   'topic'
	dhibee namichaa                'suffering/undergoing'
	meeshaa manaa                  'containing'
	huccuu bardheengaddaa          'the time of'
	foddaa gamoo                   'constituent part
	charjerii komputeraa           'associated part'
	madda odeeffannoo              'cause'
	miidhaa waraanaa               'result'
	
	
	The genitive determiner and pronoun system in Afaan Oromoo:
	
	Person, Gender, Number           Possessive Determiner             Possessive Pronoun
	1st person singular
	(all genders)                    koo (kitaabakoo)                       koo (kun kooti)
	
	1st person plural
	(all genders)                    keenya (kitaabakeenya)                keenya (kun keenyadha)
	
	2nd person singuar
	(all genders)                   kee (kitaaba kee)                     kee (kun  kitaabakeeti)
	
	2nd person plural
	(all genders)                  keessan (kitaaba keessan)              keessan (kun kitaaba keessanidha)
	
	3rd person singular
	masculine
	(all dialects                 isaa (ktaabasaa)                        isaa (kun kitaabasaati)
	
	3rd person singular
	feminine
	(all dialects)             ishee (kitaabashee)                       ishee (kun kitaabashee)
	
	3rd person plural
	all genders               isaanii (kitaabasaanii)                    isaanii (kun kitaabasaaniiti)
	
	3rd person singular
	neuter                    ---------                                 ----------------
	
	\section{Proposed Solution to Genitive of Genitive Problems in Afaan Oromoo}
	
	
	\section{Conclusion}
	
	\newpage
	\bibliographystyle{apacite}
	\bibliography{genitive}
	
	\newpage
	\section*{Acronyms and symbols}
	c=palatal,\\
	plosive, affricate\\
	GEN=genitive\\
	MID=middle\\
	pl=plural\\
	sing=singular\\
	
\end{document}
