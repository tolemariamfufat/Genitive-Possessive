\documentclass[11pt,a4paper]{article}
\usepackage{linguex}
\usepackage{tipa}
\usepackage{apacite,pslatex}
\usepackage{titling}
\title {Genitive Construction and Dissimilation in Afaan Oromoo}
\author {Tolemariam Fufa}
\date{October 2022}
\begin{document}
	\newcommand\keywords[1]{%
		\begingroup
		\let\and\\
		\par
		\noindent\textbf{Keywords:}\\#1\par
		\endgroup
	}
	\maketitle
	\begin{abstract}
		 
		This paper attempts to describe Afaan Oromoo genitive construction.	This paper is divided into 6 sections. The first section is an introduction to Afaan Oromoo genitive construction. Section 2 is discusses possessive genitives of Afaan Oromoo; the concern of this section is alienable possessive (ALP) and inalienable possessive (IAP) of Afaan Oromoo. Section three discusses descriptive genitive of the language. Section 4 deals with problem areas of genitive of genitives of Afaan Oromoo. Section 5 proposes solutions for problems of genitive of genitives. Section six concludes the paper.
	\end{abstract}
	\keywords{genitive, possessive, descriptive, alienable, inalienable}
	\newpage
	
	\section{Introduction}
	\label{sec:org462cb4a}
	
	The genitive construction usually associated to owenership of something.  In Afaan Oromoo the genitive case is 
	
	
	
	The genitive construction usually associated to ownership of something; it is often described as possessive case. But, close investigation shows that genitive construction includes possession, source, origin and description among other things; that is to say all issues concerning genitive case are not explainable in terms of possession. Therefore, genitive construction is treated in different ways in literature; it is often typed into various categories including possessive, subjective, source, origin, objective, partitive and descriptive genitive \cite{ELTcon}. Some researchers categorize genitive construction into two main categories; namely, possessive and descriptive genitive; these two main categories are further subdivided into subcategories as alienable and inalienable possessions, metaphorical, generic descriptive genitives \cite{rosenbach2006descriptive,gebregziabher2012alienable}. 
	
	
	The discussion of subjective, objective, source, material and partitive genitive is also common. Subjective and objective genitives are used when a noun is derived from a verb and modified by another dependent noun. If the dependent noun (modifier noun of the main noun) expresses the main noun that is derived from the original verb, it is claimed to be a subjective genitive; but, if the dependent noun (genitive) expresses the object of the main verb, the genitive is said to be objective genitive. In subjective genitive an action or idea is set forth as proceeding from the noun in subject position \cite[68]{greenlee1950genitive}. The subjective genitive expresses about the nature of subject of a sentence. It does’t refer to possession. 
	
	
	In Afaan Oromoo subjective and objective genitives cannot be distinguished morphologically or syntactically; they are expressed in the same with construction. Thus, subjective and objective ginitives are distinguished semantically. Therefore, context is the main determining factor in this case. \\
	
	\ex. 
	\ag.
	nam-ni isa jaal-at-a\\
	man(people)-NOM him love-MID-3MSS\\
	'People loves him.'\\
	
	\ex. 
	\ag.
	inni nama jaal-at-a\\
	he.NOM man(people) love-MID-3MSS\\
	'He loves people.'\\
	
	\ex. 
	\ag.
	jaalala nama-a\\
	love man(people)-GEN\\
	'People's love or love of people.'\\
	
	As shown above \emph{jaalala} 'love' is derived from the verb \emph{jaalata} 'loves'. The dependent is \emph{nama} 'people' in this context. The dependent \emph{nama} express the subjective genitive on the bases of example (1) provided that people loves him. If we take into consideration example (2), \emph{jaalala namaa} 'people's love' is also subjective genitive as long as \emph{inni} 'he' dedicated to welfare of humanity. But if we are talking about general observation love of human being for human being, \emph{jaalala namaa} 'love of people' becomes objective genitive. 
	
	
	Moreover, a source can be expressed by a genitive case \cite[69]{greenlee1950genitive}. \\
	
	\ex.
	\ag.
	caama bona-a\\
	drought summar-GEN\\
	'summer drought'\\
	
	\ex.
	\ag.
	lolaa ganna-a\\
	flood winter-GEN\\
	'winter flood'\\
	
	In (4) \emph{bona} 'summer' is considered to be the source of drought; and in (5) \emph{ganna} 'Winter' happens to be the source of flood. In both cases genitive is marked by lengthening the final short vowels of the corresponding nouns, \emph{bona} and \emph{ganna} respectively. 
	
	In order to clarify source genitive, let us consider the following examples:
	\ex.
	\ag.
	buna Kafaa\\
	coffee Wallaggaa.GEN\\
	'Kafaa coffee'\\
	
	\ex.
	\ag.
	k'amadii Arsii\\
	wheat Arsii.GEN\\
	'Arsii Wheat'\\
	
	\ex.
	\ag.
	hoolaa Wolayitaa\\
	sheep Harar.GEN\\
	'Wolayita sheep'\\
	
	Examples given in (6-8) are source genitives. (6) expresses coffee is from \emph{Kafaa}; the sources of wheat and bull are also expressed in the same way, namely from \emph{Arsii} and \emph{Wolayitaa} respectively. Because the dependent nouns in (6-8) have long final vowels, the genitive is marked by null morpheme. 
	
	Materials of which a thing is made can be expressed by the genitive case \cite[69]{greenlee1950genitive}.
	\ex.
	\ag.
	siree sibiilaa\\
	bed iron.GEN\\
	'iron bed'\\
	
	\ex.
	\ag.
	foon hoolaa\\
	meat sheep.GEN\\
	'mutton'\\
	
	\ex.
	\ag.
	mana \textipa{\!d}agaa\\
	house stone.GEN\\
	'stone house'\\
	
	In (9), the material from which a bed is made, in (10) a type of meat and in (11), a material from which a house is built are expressed by genitive \emph{sibiilaa} 'iron', \emph{hoolaa} 'sheep' and \emph{\textipa{\!d}agaa} 'stone' respectively.
	
	Not only materials of which a thing is made but also a purpose a material for which it is planned my be expressed by genitive. In this case word order is the reverse of examples given in (9-11).
	
	\ex.
	\ag.
	siibiila siree\\
	iron bed.GEN\\
	'bed iron'\\
	
	\ex.
	\ag.
	foon hoolaa\\
	meat sheep.GEN\\
	'meat sheep'\\
	
	\ex.
	\ag.
	\textipa{\!d}agaa manaa\\
	stone house.GEN\\
	'house stone'\\
	
	As shown above word order of genitive phrases given in (12-14) are the reverse of examples given in (9-11) to show that purpose and material genitives are expressed not only morphologically but also syntactically.
	
	A whole thing from which only a part is to be focused may expressed by genitive construction known as the partitive genitive \cite[:69]{greenlee1950genitive}. Although most partitive 
	constructions refer to a quantity or amount, some are used to indicate quality or behavior.  
	\ex.
	\ag.
	halkan walakkaa\\
	night half.GEN\\
	'middle of the night'\\
	
	\ex.
	\ag.
	lit'a gannaa \\
	beginning winter.GEN\\
	'beginning of winter'\\
	
	\ex.
	\ag.
	baha gannaa\\
	end winter.GEN\\
	'end of winter'\\
	
	\ex.
	\ag.
	barii Dilbataa\\
	dawn Sunday.GEN\\
	'dawn of the Sunday'\\
	
	The above given examples indicate time. In (15), \emph{halkan walakkaa} 'middle of the night' indicates not only amount of time (which is roughly 6 hours) but also quality or behavior linked to middle of the night which is silent and fearsome. Similarly, (16-18) indicate amount of time as well as behaviors associated with \emph{lit'a gannaa} 'beginning of winter', \emph{baha gannaa} 'beginning of winter' and \emph{barii Dilbataa} 'dawn of the Sunday'. In the beginning of winter farmers prepare themselves for farming season. By contrast end of winter is associated with new year celebrations and time of happiness and joy. Dawn of the Sunday is also linked with time of rest, worship and socialization. 
		
	In Afaan Oromoo, genitive constructions marked in six ways:\\
	
	(1) A noun ending in long vowel marked by null morpheme. For example,  in \emph{mana Jiraataa} 'Jiraataa's house' the dependent (possessor) \emph{Jiraataa} has long final vowel; therefore, there is no overt genitive marking in such case. That is to say the possessor \emph{Jiraataa} doesn't show any morphological change to indicate genitive construction. \\
	
	(2) If a noun ends with a short vowel, we mark genitive case by lengthening the short vowel \cite[183]{gragg1976oromo,gobena2019verb}. For example, in \emph{mana namaa} 'a man's house', \emph{nama} 'man' has final short vowel in absolute form; the final short vowel becomes long in genitive construction. \\
	
	(3) If a noun ends in consonant, we mark the genitive construction by suffixation of \emph{-ii} \cite{gobena2019verb}. For example in \emph{wark'ii ilkaanii} 'golden teeth' the noun \emph{ilkaan} 'teeth' ends with \emph{n}, to which the genitive mark \emph{-ii} is suffixed. A noun that got its vowel long is interpreted as the possessor of a noun that precedes it. \\
	
	(4) The genitive can be made by placing the pronoun \emph{kan} 'whose' between the head and dependent (possessor) nouns; for example \emph{mana kan k'ottuu} 'a farmers's house' \cite[183]{gragg1976oromo}. In Meca dialect the pronoun \emph{kan} is used for both masculine and feminine genders. But in \emph{Harar, Borena and Tullama} dialects \emph{kan} shows masculine while \emph{tan} shows feminine gender. For example, \emph{mana tan ishee} 'her house'. \\
	
	(5) Genitves of genitives which are not final retain their absolute form \cite[104]{owens1985grammar}. \\
	
	(6) By applying dissimilation rule to genitive of genitives;
	
	\ex.
	\ag.
	mana nama-a \\
	house man-GEN\\
	'a man's house'\\
	
	\ex.
	\ag.
	mana nama Jimmaa \\
	house man Jima.GEN\\
	'Jima man's house'\\
	
	(20) is a genitive of (19). In (19), the dependent noun \emph{namaa} 'man' has final long vowel to mark genitive construction. In genitive of a genitive shown in (20), \emph{namaa} loses its final long vowel where another namely, \emph{Jimmaa} becomes the second dependent.
	
	(6) Genitive construction can also be indicated by possessive pronouns. For example,\\
	\\
	\ex.
	\ag.
	mana koo \\
	house my \\
	'my house'\\
	
	\ex.
	\ag.
	mana kee \\
	house your\\
	'your house (sing)'\\
	
	\ex.
	\ag.
	mana ishee \\
	house her\\
	'her house'\\
	
	\ex.
	\ag.
	mana isaa \\
	house his\\
	'his house'\\
	
	\ex.
	\ag.
	mana keenya \\
	house our\\
	'our house'\\
	
	\ex.
	\ag.
	mana keessan \\
	house your\\
	'your house (pl)'\\

	
	In this paper I categorize genitive construction into two types: possessive and descriptive genitives. Descriptive genitives are genitive constructions which are not possessives \cite{rosenbach2006descriptive}.
	
	\section{Possessive Genitives}
	
	Genitive construction often employed to express possession \cite[68]{greenlee1950genitive}.
	In Afaan Oromoo, possessive genitive can be expressed either morphologically or syntactically. 
	
	\ex.
	\ag.
	mana Tolasaa\\
	house Tolasa.GEN\\
	'Tolasa's house'
	
	\ex.
	\ag.
	Tolasaa-n mana qab-a\\
	Tolasa-NOM house has-3MSS\\
	'Tolasa has a house.'
	
	\ex.
	\ag.
	obboleessa Boontuu\\
	brother Bontu.GEN
	'Bontu's brother'\\
		
	\ex.
	\ag.
	Boontuu-n obboleessa qab-di\\
	Bontu-NOM brother has-3FSS\\
	'Bontu has brother.'
	
	
	As shown in the above examples possessive genitives in (27) and (29) are expressed morphologically while (28) and (30) are expressed syntactically using verb to have \emph{qaba/qabdi}. 
	
	Possessive genitive is categorized into two types: alienable (APG) and inalienable possessions (APG) \cite{gebregziabher2012alienable}.
	
	\subsection{Alienable Possessive Genitive (APG)}
	
	Alienable possessions refers to possessions which have not fixed semantic relationships between the 
	possessor and the possessed nouns \cite{gebregziabher2012alienable}. That is to say alienable possessions can freely change ownership. These include materials such as car, house, computer, book, etc.
	
	\ex.
	\ag.
	kitaaba barataa\\
	book student.GEN\\
	'student's book'
	
	\ex.
	\ag.
	konkolaataa barsiisaa\\
	car teacher.GEN\\
	'teacher's car'

	\ex.
	\ag.
	bilbila shamarree\\
	phone girl.GEN\\
	'girl's phone'
	
	
	These examples indicate morphological APG. APG can be expressed syntactically by employing the verb to have \emph{qab-}. 
	
		
	\ex.
	\ag.
	Barataa-n kitaaba qab-a\\
	student-NOM book has-3MSS\\
	'A student has a book.'
	
	\ex.
	\ag.
	Barsiisaa-n konkolaataa qab-a\\
	teacher-NOM car has-3MSS\\
	'A teacher has car.'
	
	\ex.
	\ag.
	shamarree-n bilbila qab-di\\
	girl-NOM phone has-3FSS\\
	'A girl has phone.'
	
	These are genitive constructions expressed syntactically by employing the verb to have \emph{qaba/qabdi}.
	
	Concerning definite morpheme,  APG is symmetrical with analytical (syntactic) APG. For example, the definite marker can get attached to the main noun in morphological APG. 
	
	\ex.
	\ag.
	kitaab-icha barataa\\
	book-DEF student.GEN\\
	'a student's book/lit. a student's the book'
	
	\ex.
	\ag.
	konkolaat-icha barsiisaa\\
	car-DEF teacher.GEN\\
	'a teacher's car/lit. a teacher's the car'
	
	\ex.
	\ag.
	bilbil-icha shamarree\\
	phoone-DEF girl.GEN\\
	'a girl's phone/lit. a girl's the phone'
	
	In the above examples all the head nouns are marked for definite; and, all are grammatically correct. 
	
	The dependent noun in the possessive genitive can also be marked for definiteness. 
	
	\ex.
	\ag.
	kitaaba barat-ichaa\\
	book-DEF student.GEN.DEF\\
	'the student's book'
	
	Both the head noun and the dependent nouns may also be marked for definiteness in Afaan Oromoo.
	
	\ex.
	\ag.
	bilbil-icha shamarr-ittii\\
	phoone-DEF girl.GEN.DEF\\
	'the girl's phone/lit. the girl's the phone'
	
	Similarly, analytic APG allows for the complement of the verb to have to be marked for definite. 
	
	\ex.
	\ag.
	barataa-n kitaab-icha k'ab-a\\
	student-NOM book-DEF has-3MSS\\
	'A student has the book.'
	
	\ex.
	\ag.
	barsiisaa-n konkolaat-icha k'ab-a\\
	teacher-NOM car-DEF has-3MSS\\
	'A teacher has the car.'
	
	\ex.
	\ag.
	shamarree-n bilbil-icha k'ab-di\\
	girl-NOM phone-DEF has-3FSS\\
	'A girl has the phone.'
	
	As shown above head nouns of genitive construction which are complements of the verb to have permits  for the definite morpheme \emph{-icha} to get attached to theme.
	
	
	The head noun of the APG can be a genitive construction. 
	
	\ex.
	\ag.
	kitaaba herreega-a barataa\\
	book maths-GEN student.GEN\\
	'student's maths book'\\

	\ex.
	\ag.
	barataan kitaaba herreegaa qab-a\\
	student book  maths.GEN has-3MSS\\
	'A student has maths book.'\\
	
	As shown above  in (43) and (44) the head noun is a genitive construction. In genitive structure such as shown in (43), \emph{kan} 'of' can be added. 
	
	\ex.
	\ag.
	kitaaba kan barataa\\
	book of student.GEN\\
	'student's book/book of a student'\\
	
	\ex.
	\ag.
	kitaaba herreega-a kan barataa\\
	book maths-GEN of student.GEN\\
	'student's maths book/maths book of a student'\\
	
	
	
	The word \emph{kan} is added after the head noun. In the analytic structure shown below the word \emph{kan} 'of' can be optionally added. In this case it is not 'student's book of maths' but 'book of maths' as shown below.
	
	\ex.
	\ag.
	barataan kitaaba kan herreegaa qab-a\\
	student book  of maths.GEN has-3MSS\\
	'A student has a book of maths.'\\
	
	If the head noun is modified by an adjective or a quantifier, the morphological ALP need the insertion of \emph{kan} 'of' to be grammatical while the analytic structure of ALP doesn't. 
	
	\ex.
	\ag.
	*kitaaba guddaa barataa\\
	book big student.GEN\\
	'student's big book'\\
	
	\ex.
	\ag.
	kitaaba guddaa kan barataa\\
	book big of student.GEN\\
	'student's big book'\\
	
	\ex.
	\ag.
	*kitaab-ilee lama barataa\\
	book-PL two student.GEN\\
	'student's two books'\\
	
	\ex.
	\ag.
	kitaab-ilee lama kan barataa\\
	book-PL two of student.GEN\\
	'student's two books'\\
	
	(47) and (49) are ungrammatical because the head nouns of the possessive genitives are modified by an adjective and a quantifier. But, (48) and (50) are grammatical because the head nouns of the genitives are modified and the word \emph{kan} 'of' is inserted preceding the dependent nouns. 
	
	By contrast, the analytic APG is grammatical if modified by an adjective or a quantifier without an addition of the word \emph{kan} 'of' as shown below.
	
	\ex.
	\ag.
	barataa-n kitaaba guddaa k'ab-a\\
	student-NOM book big has-3MSS\\
	'A student has a big book.'
	
	\ex.
	\ag.
	barataa-n kitaab-ilee lama k'ab-a\\
	student-NOM book-PL two has-3MSS\\
	'A student has two books.'
	
	As far as morphological ALP is concerned the more modifier stuff is added, the less the genitive construction becomes transparent and the less grammatical the structure would be. 
	
	\ex.
	\ag.
	*kitaaba herreega-a guddaa barataa\\
	book maths-GEN of big student.GEN\\
	'student's big maths book/big maths book of a student'\\
	
	\ex.
	\ag.
	kitaaba herreega-a guddaa kan barataa\\
	book maths-GEN of big of student.GEN\\
	'student's big maths book/big maths book of a student'\\
	
	\ex.
	\ag.
	*kitaaba herreega-a guddaa sana barataa\\
	book maths-GEN of big that student.GEN\\
	'student's that big maths book/that big maths book of a student'\\
	
	\ex.
	\ag.
	kitaaba herreega-a guddaa sana kan barataa\\
	book maths-GEN of big that of student.GEN\\
	'student's that big maths book/that big maths book of a student'\\
	
	As shown above (56) and (58) are less acceptable because in (56) an adjective \emph{guddaa} 'big' is added to the already heavy main noun \emph{kitaaba herreegaa} 'maths book' as a result of which the meaning of ALP becomes less transparent. The same case is true with (58) where an adjective \emph{guddaa} 'big' and a demonstrative \emph{san} 'that' are added. But if the word \emph{kan} 'of' is inserted preceding the dependent noun as shown in (57) and (59), both possessive genitives become acceptable. 
	
	As compared to morphological ALP analytic ones remain grammatically acceptable whether or not an adjective or a demonstrative added to modify the main noun of the genitive.
	
	\ex.
	\ag.
	barataa-n kitaaba herreegaa guddaa k'ab-a\\
	student-NOM book maths.GEN big has-3MSS\\
	'A student has a big maths book.'
	
	\ex.
	\ag.
	barataa-n kitaaba herreegaa guddaa sana k'ab-a\\
	student-NOM book maths.GEN big that has-3MSS\\
	'A student has that big maths book.'
	
	In the analytic ALP, the word \emph{kan} 'of' can be optionally inserted preceding the verb to have \emph{k'ab-} 'has'.
	
	\ex.
	\ag.
	barataa-n kitaaba kan herreegaa guddaa k'ab-a\\
	student-NOM book of maths.GEN big has-3MSS\\
	'A student has a big book of maths.'
	
	\ex.
	\ag.
	barataa-n kitaaba kan herreegaa guddaa sana k'ab-a\\
	student-NOM book of maths.GEN big that has-3MSS\\
	'A student has that big book of maths.'
	
	If the complement of the verb to have is not a genitive itself, the word \emph{kan} 'of' cannot be used in the analytic ALP.
	
	\ex.
	\ag.
	*barataa-n kitaaba kan guddaa k'ab-a\\
	student-NOM book of big has-3MSS\\
	'A student has a big book.'
	
	\ex.
	\ag.
	*barataa-n kitaaba kan guddaa sana k'ab-a\\
	student-NOM book of big that has-3MSS\\
	'A student has that big book.'
	
	
	\subsection{Inalienable Possessive Genitive (IPG)}
	
	Inalienable genitives (IPG) are possessive cases which forms a fixed semantic relationship between the possessor and the  possessee; kinship and body part possessive genitives are considered to be prototypes of IPG \cite[161]{gebregziabher2012alienable}. 
	
	\ex.
	\ag.
	abbaa gurbaa\\
	father boy.GEN\\
	'a boy's father'\\
	
	\ex.
	\ag.
	rifeensa hintala-a\\
	hair girl-GEN\\
	'a girl's hair'\\
	
	As shown in (66), the possessor or the dependent noun is \emph{gurbaa} 'boy' wile the possessee is \emph{abbaa} 'father'. In this the possessor and the possessee form fixed semantic relationship in the sense that the former is defined relative to the latter or vice versa. 
	
	Similarly, in (67) the noun \emph{rifeensa} 'hair' is defined in relation to the possessor which happens to be a specific individual. Body parts inherently require to be part of somebody \cite[161]{gebregziabher2012alienable}.
	
	In Afaan Oromoo IPG is expressed in the same as APG be it morphological or syntactic as shown below. \\
	
	APG
	
	\ex.
	\ag.
	mana hintala-a\\
	house girl-GEN\\
	'a girl's house'\\
	
	\ex.
	\ag.
	hintall-i mana k'ab-di\\
	girl-NOM house has-3FSS\\
	'A girl has house.'\\
	
	IPG
	
	\ex.
	\ag.
	rifeensa hintala-a\\
	hair girl-GEN\\
	'a girl's hair'\\
	
	\ex.
	\ag.
	hintall-i rifeensa k'ab-di\\
	girl-NOM hair has-3FSS\\
	'A girl has hair.'\\
	
	In Afaan Oromoo, the main noun of the IPG can be modified by another IPG.
	
	\ex.
	\ag.
	abbaa gurbaa\\
	father boy.GEN\\
	'a boy's father'\\
	
	\ex.
	\ag.
	abbaa abbaa gurbaa\\
	father father.GEN boy.GEN\\
	'a boy's father's father'\\
	
	\ex.
	\ag.
	abbaa eessuma/abeeraa gurbaa\\
	father uncle.GEN boy.GEN\\
	'a boy's uncle's father'\\
	
	\ex.
	\ag.
	abbaa abbaa eessuma/abeeraa gurbaa\\
	father father.GEN uncle.GEN boy.GEN\\
	'a boy's uncle's father's father'\\
	
	As shown in (73), the head noun of IPG is modified by another similar noun \emph{abbaa} 'father'. In (75), the head noun of the possessive genitive is modified by another possessive genitive, namely \emph{abbaa eessuma/abeeraa} 'uncle's father'. 
	
	The above extended IPG can be expressed syntactically as shown below.
	
	\ex.
	\ag.
	gurbaa-n abbaa k'ab-a\\
	boy-NOM father has-3MSS\\
	'A boy has father.'\\
	
	\ex.
	\ag.
	gurbaa-n abbaa abbaa k'ab-a\\
	boy-NOM father father.GEN has-3MSS\\
	'A boy has father's father.'\\
	
	\ex.
	\ag.
	gurbaa-n abbaa eessuma/abeeraa k'ab-a\\
	boy-NOM father uncle.GEN has-3MSS\\
	'A boy has uncle's father.'\\
	
	\ex.
	\ag.
	gurbaa-n abbaa abbaa eessuma/abeeraa k'ab-a\\
	boy-NOM father father.GEN uncle.GEN has-3MSS\\
	'A boy has uncle's father's father.'\\
	
	It is natural to modify the dependent noun (possessor) in IPG.
		
	\ex.
	\ag.
	abbaa gurb-icha-a\\
	father boy-DEF-GEN\\
	'the boy's father'\\
	
	\ex.
	\ag.
	rifeensa hintal-ittii\\
	hair girl-DEF.GEN\\
	'the girl's hair'\\
	
	In IPG, modifying the head noun by a determiner is less acceptable in Afaan Oromoo.
	
	\ex.
	\ag.
	*abb-icha gurbaa\\
	father-DEF boy.GEN\\
	'a boy's the father'\\
	
	\ex.
	\ag.
	*rifeens-icha hintala-a\\
	hair-DEF girl.GEN\\
	'a girl's the hair'\\
	
	\ex.
	\ag.
	*abb-icha gurb-icha-a\\
	father-DEF boy-DEF-GEN\\
	'the boy's the father'\\
	
	\ex.
	\ag.
	*rifeens-icha hintal-ittii\\
	hair-DEF girl-DEF.GEN\\
	'the girl's the hair'\\
	
	In (83) and (83) head nouns of the corresponding alienable possessives are modified by the definite marker \emph{-icha} 'the'. In both cases alienable possessives are less acceptable. In (84) and (85) both the head and dependent nouns are modified by definite markers \emph{-icha, ittii}; yet, the phrases fail to be alienable possessive construction. 
	
	In Afaan Oromoo, the possessor noun can be modified by an adjective and definite marker.
	
	\ex.
	\ag.
	obboleessa   gurbaa diim-aa\\
	brother 	boy red-M.GEN\\
	'a red boy's brother'\\
	
	\ex.
	\ag.
	harka hintala diim-tuu\\
	hand girl red-F.GEN\\
	'a red girl's hand'\\
	
	\ex.
	\ag.
	gurbaa diim-aa-n obboleessa k'ab-a\\
	boy    red-M-NOM brother has-3MSS\\
	'A red boy has brother.'\\
	
	\ex.
	\ag.
	*hintala diim-tuu-n harka k'ab-di\\
	girl red-F-NOM hand has-3FSS\
	'A red girl has hand.'\\
	
	\ex.
	\ag.
	hintala diim-tuu-n harka bareed-duu k'ab-di\\
	girl red-F-NOM hand beautiful-F has-3FSS\\
	'A red girl has a beautiful hand.'\\
	
	As shown in (86) and (87) dependent nouns are modified by adjectives. But as shown in (89) head noun of a body part cannot be expressed analytically unless modified by an adjective while kinship possessive genitives can be expressed analytically without being modified by adjective as shown in (88). 
	
	Moreover, the possessor in IPG can be modified by an adjective, definite marker and demonstrative pronouns.
	
	\ex.
	\ag.
	obboleessa   gurb-icha diim-aa\\
	brother 	boy-DEF red-M.GEN\\
	'the red boy's brother'\\
	
	\ex.
	\ag.
	obboleessa   gurb-icha diim-aa sana-a\\
	brother 	boy-DEF red-M that-GEN\\
	'that (the) red boy's brother'\\
	
	\ex.
	\ag.
	harka hintal-ittii diim-tuu\\
	hand girl-DEF red-F.GEN\\
	'the red girl's hand'\\
	
	\ex.
	\ag.
	harka hintal-ittii diim-tuu sana-a\\
	hand girl-DEF red-F that-GEN\\
	'that (the) red girl's hand'\\
	
	In (90) and (92) the dependent noun or the possessors are modified both by adjective and definite marker; and, in (91) and (93) the head nouns are modified by adjective, definite marker and demonstrative. 
	
	IPG can be expressed both analytically and morphologically. Therefore, analytic counterparts can be shown below.
	
	\ex.
	\ag.
	gurb-ich-i diim-aa-n obboleessa k'ab-a\\
	boy-DEF-NOM red-M-NOM brother has-3MSS\\
	'The red boy has brother'\\
	
	\ex.
	\ag.
	gurb-ich-i diim-aa-n sun obboleessa k'ab-a\\
	boy-DEF-NOM red-M-NOM that brother has-3MSS\\
	'That (the) red boy has brother'\\
	
	\ex.
	\ag.
	hintal-ittii diim-tuu-n harka bareed-duu k'ab-di\\
	girl-DEF.F red-F-NOM hand beauitful-F has-3FSS\\
	'The red girl has a beauitful  hand.'\\
	
		
	\ex.
	\ag.
	hintal-ittii diim-tuu-n sun  harka bareed-duu k'ab-di\\
	girl-DEF.F red-F-NOM that hand beautiful-F has-3FSS\\
	'That (the) red girl has a beautiful  hand.'\\
	
	
	
	
	\section{Descriptive Genitive}
	
	
	
	\begin{enumerate}
		\item how do descriptive genitives differ from possessive genitives?
		\item Are descriptive genitives syntactic, morphological or compounds?
		\item How do descriptive genitives differ from N + N sequences?
	\end{enumerate}
	
	Descriptive genitive is not an NP but usually a noun. Genitive constructions in which the possessor functions as a determiner have NP status and they denote a specific entity.Possessive genetives expands nominals into noun phrases. Semantically, possessive genitives specify (in)definiteness and establish reference within the NP \cite[82]{rosenbach2006descriptive}. In contrast the dependant in descriptive genitives is not an NP but usually a noun. The semantic difference between determiner genitives and descriptive genitives as discussed above are reflected in different positions in Afaan Oromoo noun phrases. Word order in the noun phrase is iconically determined in that any element contributing to the denotation of the head noun is positioned adjacent to the head, while anything contributing to the reference of the noun phrase will be most distantly located away from the head noun \cite[81]{rosenbach2006descriptive}.
	
	Therefore descriptive genitives are themeselves not full NPs but nouns or nominals and, in contrast to 	determine genitives, they denote properties and not specific entities. 	Semantically, the dependent in descriptive genitives contributes to the denotation of the head noun, not to the spacification of it. The dependant noun has a classifiying function in such genitives. As a classifier, the dependent is not referential and does not refer to a specific referent. 
	
	In Afaan Oromoo, the head can be separately determined by definite article or by other reference tracking devices in possessive genitives.
	
	\ex.
	\ag.
	mana citaa \\
	house straw.GEN\\
	'a straw house'\\
	
	\ex.
	\ag.
	mana-icha citaa \\
	house-DEF straw.GEN\\
	'the straw house'\\
	
	\ex.
	\ag.
	mana cit-icha-a \\
	house straw-DEF-GEN\\
	'the straw house'\\
	
	\ex.
	\ag.
	*mana-icha cit-icha-a \\
	house-DEF straw-DEF-GEN\\
	'the straw house'\\
	
	(99) is a descriptive genitive where both the head and dependent nouns are not modified by definite article. (100) and are also acceptable; in these phrases either the head or dependent nouns are modified by definite article. (102) is less acceptable may be because both head and dependent are modified in the same phrase. 
		
	\ex.
	\ag.
	mana nama-a\\
	house mana-GEN\\
	'Someone's house/lit. a mana's house'\\
	

	\ex.
	\ag.
	mana nam-icha-a\\
	house mana-DEF-GEN\\
	'(the) someone's house/lit. (the) mana's house'\\
	
	Semantically, the possessor \emph{namaa} 'man's' in (103) functions like the definite article, specifying the referent of the NP. In this example \emph{namaa} specifiees whose house it is, namely someone's. From a cognitive-pragmatic and semantic point of view the possessor can be viewed as an 'anchor' that narrows down the referent of the NP \cite[80]{rosenbach2006descriptive}. In Afaan Oromoo, the dependent can be modified by definite article or demonstrative pronouns as discussed earlier in APG and IPG sections.
	
	In Afaan Oromoo the possessor can be postmodified and can be headed by a final determiner and demonstratives. Descriptive genitives can also be posmodified by demonstrative.
	
	\ex.
	\ag.
	mana citaa \\
	house straw.GEN\\
	'a straw house'\\
	
	\ex.
	\ag.
	mana citaa sana \\
	house straw.GEN that\\
	'that  straw house'\\
	
	\ex.
	\ag.
	mana nam-icha sana-a\\
	house mana-DEF that-GEN\\
	'that (the) someone's house/lit. that (the) mana's house'\\
	
	As shown in (106), the demonstrative \emph{sana} 'that' has short final vowel; it is not marked for genitive case; it is the dependent \emph{citaa} 'straw's' that is marked for genitive case to show that the dependent is not NP, but N. But the dependent in (107) is NP because the demonstrative \emph{sanaa} is marked for genitive case while the dependent \emph{citaa} 'straw' is not; this shows that the dependent in possessive genitive is an NP. 
	
	\subsection{Are Descriptive Genitives Syntactic Phrases or Compounds?}
		
	To know Whether or not descriptive genitives are syntactic phrases or compounds, we need to consider three criterion. These are coordination test, modification of the dependent and head nouns \cite[83]{rosenbach2006descriptive}. 
	
	In terms of coordination test, we need to see if descriptive genitives allow a third element to be inserted between the head and dependent nouns. If a third element is allowed to be inserted between them or elements are allowed to freely move, descriptive genitives are proved to be syntactic phrases. But if a third element is not allowed to be inserted between head noun and possessor, or constituent elements are not allowed to freely move, then they are considered to be compounds. 
	
	In Afaan Oromoo, descriptive genitives pass coordination test. Let us observe the following material descriptive genitives. 
	
	
	\ex.
	\ag.
	mana \textipa{\!d}agaa fi k'ork'oorroo\\
	house stone.GEN and iron-sheet.GEN\\
	'stone and iron sheet house'\\
	
	\ex.
	\ag.
	mana citaa fi \textipa{\!d}ok'ee  \\
	house straw.GEN and mud.GEN\\
	'straw and mud house'\\

	\ex.
	\ag.
	aannan sa\textipa{\textbarglotstop}a-a fi gaala-a\\
	milk cow-GEN and camel-GEN\\
	'goat and camel milk'\\
	
	In (105) and (106) dependent nouns indicate materials from which head nouns are built; that is in (105) a house is built from stone and iron sheet while in (106) a house is built from straw and mud. In (107) milk is produced from goat and camel. 
	
	To make clear that descriptive genitives are not compounds we can further test if material descriptive genitives shown above can be converted into purpose genitives as shown below.
	
	\ex.
	\ag.
	\textipa{\!d}agaa fi k'ork'oorroo mana-a\\
	stone and iron-sheet house-GEN\\
	'stone and iron sheet for house'\\
	
	\ex.
	\ag.
	citaa fi \textipa{\!d}ok'ee mana-a \\
	straw and mud house-GEN\\
	'straw and mud for house'\\
	
	\ex.
	\ag.
	sa\textipa{\textbarglotstop}a fi gaala aannan-ii\\
	cow and camel milk-GEN\\
	'goat and camel for milk'\\
	
	As shown above constituent elements of descriptive genitives freely move to show that they are not compounds but syntactic phrases. 
	
	The second test whether or not the dependent is separately modified by definite article. If an N + N genitive construction is a compound, then it should not be possible to separately modify the dependent. But in Afaan Oromoo dependent in the N +N can be modified as shown below.
	
	
	\ex.
	\ag.
	mana citaa \\
	house straw.GEN\\
	'straw house'\\
	
	\ex.
	\ag.
	mana cit-icha-a \\
	house straw-DEF-GEN\\
	'house of the straw'\\
	
	\ex.
	\ag.
	aannan gaala-a \\
	milk camel-GEN\\
	'camel's milk'\\
	
	\ex.
	\ag.
	aannan gaal-icha-a \\
	milk camel-DEF-GEN\\
	'milk of the camel'\\
	
	The ability to intervene between the dependent and the head noun is the strongest test whether or not descriptive genitive is a compound or syntactic phrase \cite[85]{rosenbach2006descriptive}. In Afaan Oromoo, the definite article can intervene between the head and dependent nouns as shown in the following examples. 
	
	\ex.
	\ag.
	foon sangaa \\
	meat ox-GEN\\
	'ox's meat'\\
	
	\ex.
	\ag.
	foon-icha sangaa \\
	meat-DEF ox-GEN\\
	'the meat of ox'\\
	
	\ex.
	\ag.
	sangaa foon-ii\\
	ox meat-GEN\\
	'meat's ox'\\
	
	\ex.
	\ag.
	sang-icha foon-ii\\
	ox-DEF meat-GEN\\
	'the ox of meat'\\

	In the above examples not only the head and dependent exchanged orders but also heads are modified by definite articles; thus intervening between the head and dependent nouns. These examples proves that descriptive genitives are syntactic phrases, not compounds. 
	
	\subsection{Genitive of Genitives}
	
	
	The focus of this section is genitive of genitive with a special emphasis on descriptive ones. Descriptive genitive of genitive construction in this language is a new development. I think it is relevant to give a brief background of current development of Afaan Oromoo. Although Afaan Oromoo is an important language of wider communication in Ethiopia, it has’t been given 
	a chance to develop until very recently \cite[326]{bulcha1997politics}. Since the fall of the Derg regime in 1991, prohibition on the use of Afaan Oromoo has been 
	legally removed. Afaan Oromoo has got its official recognition altogether with other languages of Ethiopian nations and nationalities.  Consequently, 
	Afaan Oromoo has become a fast growing language in Ethiopia. It has become the official working language of Oromiya Regional State. Afaan Oromoo has been 
	promoted to language of mass media and medium of instruction from grade 1-8. Afaan Oromoo is taught as a subject in high schools. This indigenous language 
	is a medium of instruction and given as a field of study in Teacher’s Training Colleges in Oromiya Regional State. Teaching Afaan 	Oromoo in Ethiopian Higher Education has started. Currently, Afaan Oromoo is offered as a field of study for BA, MA and PhD level at different 
	universities of Ethiopia. Afaan Oromoo is studied as academic discipline in many Ethiopian government universities; namely Addis Ababa University, 
	Jimma University, Haromaya University, Adama Universitiy, Wallaga University, Dilla University, Mettu University, etc. 
	
	One of the visible positive effect that the change in language policy of Afaan Oromoo brought about is a high demand of naming different bureaus, places, institutions, schools, universities, cities, professions, etc of Oromiya Regional State. This naming demand brought unprecedented construction and use of  descriptive genitive of genitives. In this paper I focus on naming patterns of bureaus of Oromiya Regional State. 
	
	In Oromiya Regional State, we observe different constructions of genitive of genitives. The descriptive genitive can be simple or complex depending on the number of dependent nouns in the noun pharse. A genitive of genitive with two dependents is considered to be simple whereas a structure with more than two dependents is considered to be complex. The more the number of dependents in the construction, the more complicated would be form and meaning of the genitive structure. 
	
	The following genitive of genitives are simple.
	
	\ex.
	\ag.
	Biiroo Barnoota-a Oromiyaa\\
	bureau education-GEN Oromiya.GEN\\
	'Oromiya Education Bureau'\\
	
	\ex.
	\ag.
	Biiroo K'onna-a Oromiyaa\\
	bureau agriculture-GEN Oromiya.GEN\\
	'Oromiya Agriculture Bureau'\\
	
	\ex.
	\ag.
	Biiroo Daladala Oromiyaa\\
	bureau trade Oromiya.GEN\\
	'Oromiya Trade Bureau'\\
	
	As shown in (122-124) the head noun is \emph{biiroo} 'bureau'. All instances have two dependent nouns each. In all cases the second dependent \emph{Oromiyaa} is farther from the head noun than the second dependent noun. The first dependent describes the head noun while the second dependent describes the noun phrase \emph{biiroo barnootaa} 'education bureau' for example.
	
	The following is more complex than the above given examples.
	
	\ex.
	\ag.
	eegumsa fayyaa\\
	security health\\
	'health security\\
	
	\ex.
	\ag.
	Biiroo Eegumsa Fayyaa\\
	bureau security health.GEN\\
	'Health bureau'\\
	
	\ex.
	\ag.
	Biiroo Eegumsa Fayyaa Oromiyaa\\
	bureau security health.GEN Oromiya.GEN\\
	'Oromiya Health bureau'\\
	
	As shown above (125) is a descriptive genitive by itself where the dependent \emph{fayyaa} 'health' describes the head \emph{eegumsa} 'security'. In (126), the descriptive genitive shown in (125) modifies the head \emph{biiroo} 'bureau'. In this example, we the NP modifies the head noun \emph{biiroo} 'bureau' because \emph{eegumsa} 'security' doesn't form an NP with the head noun as separated from \emph{fayyaa} 'health'. Although the head noun is followed by three nouns, it has only two modifier dependents, \emph{eegumsa fayyaa} and \emph{Oromiyaa}. 
	
	The following example is more complex than (127).
	
	 \ex.
	 \ag.
	 Interpiraayizii Ijaarsa Hojii-wwan Bishaan Oromiyaa\\
	 enterprise construction work-PL water Oromiya.GEN\\
	 'Oromiyaa Water Works Construction Enterprise'\\
	
	
	We can break this genitive nouns phrase into its component parts.
	
	\ex.
	\ag.
	Interpiraayizii\\
	enterprise c\\
	'Enterprise'\\
	
	\ex.
	\ag.
	Hojii-wwan Bishaan-ii\\
	work-PL Water-GEN\\
	'Water Works'\\
	
	\ex.
	\ag.
	Ijaarsa Hojii-wwan Bishaan\\
	construction work-PL water Oromiya.GEN\\
	'Water Works Construction'\\
	
	\ex.
	\ag.
	Ijaarsa Hojii-wwan Bishaan Oromiyaa\\
	construction work-PL water Oromiya.GEN\\
	'Oromiya Water Works Construction'\\

	This genitive of has head noun \emph{interpiraayizii} 'enterprise'. The first dependent noun that follows the head noun is \emph{ijaarsa} 'construction' which is followed by the noun phrase dependent \emph{hojiiwwan bishaanii} 'water works'. The last dependent noun is of course \emph{Oromiya}. 
	
	\subsection{Dissimilation Rule and Economy in Genitive of Genitives}
	
	In this section we discuss rules of descriptive genitive of genitive construction in Afaan Oromoo. In Afaan Oromoo if a genitive is modified by a genitive the preceding genitive morpheme is removed if the preceding dependent has final short vowel. If the preceding dependent noun has final long vowel, dissimilation rule doesn't apply. The dissimilation rule is triggered by economy of derivation. The evidence for the dissimilation rule comes from three facts: construction of possessive genitives, causative verb derivation and inflectional suffixes such as plural markers. Before we get into detail discussion of this points, let us identify the problem. 
	
	The problem is that descriptive genitive of genitive construction is not rule based. Specifically, the place of the genitive marker is not consistent as shown below.
	
	\ex.
	\ag.
	Biiroo Barnoota-a Oromiyaa\\
	bureau education-GEN Oromiya.GEN\\
	'Oromiya Education Bureau'\\
	
	\ex.
	\ag.
	Biiroo Barnoota Oromiyaa\\
	bureau education Oromiya.GEN\\
	'Oromiya Education Bureau'\\
	
	
	\ex.
	\ag.
	Biiroo Daladala Oromiyaa\\
	bureau trade Oromiya.GEN\\
	'Oromiya Trade Bureau'\\
	
	In (133), the first dependent that is \emph{barnoota-a} 'education' is marked for genitive. But in (134) and (135), the same dependent is not marked for genitive; yet, all these instances are used by different people in written texts. So, which is one is correct the first dependent with long vowel as in (133) or with short vowel as in (134) and (135)? What rule should we follow?	
	
	
	Let us closely investigate how the genitive marker behaves in the following possessive genitive formation naturally. 
	
	\ex.
	\ag.
	mana nama-a\\
	house man-GEN\\
	'someone's house'\\ 
	
	In this example, the genitive marker is attached to noun of \emph{nama} 'man'. What happens if the dependent \emph{nama} 'man' is marked for definite?
	
	\ex.
	\ag.
	mana nam-icha-a\\
	house man-DEF-GEN\\
	'The man's house'\\ 
	
	In this case the genitive marker \emph{-a} removed from noun \emph{nama} 'man' and attached to the definite morpheme \emph{-icha} 'the'. 
	
	Let us keep on constructing the phrase by adding the demonstrative \emph{sana} 'that'. 
	
	\ex.
	\ag.
	mana nam-icha sana-a\\
	house man-DEF that-GEN\\
	'(That)The man's house'\\ 
	
	\ex.
	\ag.
	*mana nam-icha-a sana-a\\
	house man-DEF-GEN that-GEN\\
	'(That)The man's house'\\ 
	
		\ex.
	\ag.
	*mana nama-a sana-a\\
	house man-GEN that-GEN\\
	'That man's house'\\ 
	
	In this case once again the genitive morpheme moves away from the definite marker final position to demonstrative \emph{sana} as a suffix. As we can understand from ungrammatical genitive phrases (139) and (140), two genitive markers are not allowed in one phrase. Because of economy of derivation, the genitive marker that is attached to the last dependent rules. 
	
	
	\ex.
	\ag.
	*Biiroo Barnoota-a Oromiyaa\\
	bureau education-GEN Oromiya.GEN\\
	'Oromiya Education Bureau'\\
	
	\ex.
	\ag.
	Biiroo Barnoota Oromiyaa\\
	bureau education Oromiya.GEN\\
	'Oromiya Education Bureau'\\
	
	
	\ex.
	\ag.
	*Biiroo Daladala-a Oromiyaa\\
	bureau trade-GEN Oromiya.GEN\\
	'Oromiya Trade Bureau'\\
	
	\ex.
	\ag.
	Biiroo Daladala-a Oromiyaa\\
	bureau trade-GEN Oromiya.GEN\\
	'Oromiya Trade Bureau'\\
	
	Based on dissimilation rule that is triggered by economy of derivation, (142) and (144) are acceptable, whereas (141) and (143) are not. 
	
	Another strong evidence for substantiation of economy of derivation and dissimilation rule comes from causative verb derivation in Afaan Oromoo. In this language more than two causative morphemes can be attached to verb root to derive double and triple causative verbs \citeA{fufa2009typology,fufatypes}. 
	
	\ex.
	\ag.
	muka gogs-e\\
	tree dry-3MSS\\
	'He dried a tree.'
	
	\ex.
	\ag.
	muka gogs-iis-e\\
	tree dry-CAUS-3MSS\\
	'He made somebody dried a tree.'\\
	
	\ex.
	\ag.
	muka gogs-is-iis-e\\
	tree dry-CAUS-CAUS-3MSS\\
	'He made somebody got someone dried a tree.'\\
	
	In (146) the causative morpheme \emph{-iis} follows the base verb \emph{gogs} 'dry'. The main point is the causative morpheme has long vowel because it is the last derivation morpheme in (146). But, if another causative morpheme is suffixed to it, the long vowel become short and consequently the final causative morpheme gets its vowel long as shown in (147). Such economy of causative derivation and dissimilation rule is consistent in Afaan Oromoo. 
	
	Another evidence comes from plural morpheme \emph{-oota/ota}. In Afaan Oromoo, if the root noun has long vowel syllable, plural morpheme \emph{-ota} is attached to it. The plural morpheme \emph{-oota} is attached to a root with short vowel syllable. This dissimilation rule consistently applicable in this language.
	
	\ex.
	\ag.
	hool-ota\\
	sheep-PL\\
	'sheep'
	
	\ex.
	\ag.
	gaang-ota\\
	mule-PL\\
	'mules'
	
	\ex.
	\ag.
	nama-oota\\
	man-PL\\
	'men'
	
	\ex.
	\ag.
	saree-oota\\
	sar-PL\\
	'dogs'
	
	In (148) and (149) \emph{-ota} is attached to noun roots \emph{hool-} and \emph{gaang-} because both have roots with long vowels. But in (150) and (151) plural morpheme with long vowel is attached to root nouns because both roots have syllables with short vowels. 
	
	Economy of derivation and dissimilation rules shown in possessive genitive constructions, cyclic causative derivations and plural formations are equally applicable on genitive of genitive constructions in Afaan Oromoo. Economy of derivation and disssimilation rule our criteria to check if naming patterns of bureaus of Oromiya Regional State are well formed or not. For example, let us reconsider the following case:
	
	\ex.
	\ag.
	Interpiraayizii Ijaarsa Hojii-wwan Bishaan Oromiyaa\\
	enterprise construction work-PL water Oromiya.GEN\\
	'Oromiyaa Water Works Construction Enterprise'\\
	
	
	
	



	
	\section{Problems Related to Genitive of Genitive in Afaan Oromoo}

\subsection{Introduction}

Many researchers agree that Afaan Oromoo is one of a widely spoken indigenous language in Africa \cite{bulcha1997politics}. This indigenous language is a lingua franca in many parts of Ethiopia, southern half of Ethiopia in particular \cite[326]{bulcha1997politics}. Although it is an important language of wider communication in Ethiopia, Afaan Oromoo has’t been given a chance to develop until very recently. Since the fall of the Derg regime in 1991, prohibition on the use of Afaan Oromoo has been legally removed. Afaan Oromoo has got its official recognition altogether with other languages of Ethiopian nations and nationalities.  Consequently, Afaan Oromoo has become a fast growing language in Ethiopia. It has become the official working language of Oromiya Regional State. Afaan Oromoo has been 
promoted to language of mass media and medium of instruction from grade 1-8. Afaan Oromoo is taught as a subject in high schools. This indigenous language is a medium of instruction and given as a field of study in Teacher’s Training Colleges in Oromiya Regional State (Disasa, 2013: 3080). Teaching Afaan 
Oromoo in Ethiopian Higher Education has started. Currently, Afaan Oromoo is offered as a field of study for BA, MA and PhD level at different universities of Ethiopia. Afaan Oromoo is studied as academic discipline in many Ethiopian government universities; namely Addis Ababa University, 
Jimma University, Haromaya University, Adama Universitiy, Wallaga University, Dilla University, Mettu University, etc. 

Because Afaan Oromoo is a working language in Oromiyaa Regional State, all government bureaus use Afaan Oromoo noun phrases for the purpose of naming their respective offices. Almost all bureaus use descriptive genitive constructions for this purpose. We can mention the following names of bureaus for instance. 




	Biiroo Bishaaniifi Inarjii Oromiyaa 'Oromiya Water and Energy Bureau'	
	Interpiraayizii Ijaarsaa Hojiiwwan Bishaan Oromiyaa 'Oromiya Water Works Construction Enterprise'	
	Biiroo Barnootaa Oromiyaa/Biiroo Barnoota Oromiyaa 'Oromiya Education Bureau'
	Biiroo Eegumsa Fayyaa Oromiyaa 'Oromiya Health Bureau'
	Abbaa Taayitaa Galliwwan Oromiyaa 'Oromiya Revenues Bureau'
	Biiroo Qonnaa Oromiyaa 'Oromiya Bureau of Agriculture'
	Biiroo Daldala Oromiyaa 'Oromiya Trade Bureau'
	Caffee Mootummaa Naannoo Oromiyaa 'Caffee' of Oromiyaa National Regional State'
	Mana Murtii Waliigalaa Oromiyaa 'Supreme Court of Oromiya'
	
	
	The Flexibility of genitive in Afaan Oromoo
	Example                           Signaling
	koomee durba amuruu               'a physical attribute'
	obboleessa koo isa hangafa        'a blood relationship'
	abbaa manaa Uumee                'a non-blood relationship'
	alaqaa barreessituu              'a hierarchical relationship'
	hiriyaa barataa                  'a social relationship'
	garee Biiftuu                    'membership'
	beeksisa dhaabaa                 'performance'
	mana Jaalataa                    'ownership'
	gadda uummataa                   'emotional state'
	ergaa abbaashee                  'origin'
	kitaaba abbaashee isa jalqabaa   'human creation'
	dhimma biyyaa                   'topic'
	dhibee namichaa                'suffering/undergoing'
	meeshaa manaa                  'containing'
	huccuu bardheengaddaa          'the time of'
	foddaa gamoo                   'constituent part
	charjerii komputeraa           'associated part'
	madda odeeffannoo              'cause'
	miidhaa waraanaa               'result'
	
	
	The genitive determiner and pronoun system in Afaan Oromoo:
	
	Person, Gender, Number           Possessive Determiner             Possessive Pronoun
	1st person singular
	(all genders)                    koo (kitaabakoo)                       koo (kun kooti)
	
	1st person plural
	(all genders)                    keenya (kitaabakeenya)                keenya (kun keenyadha)
	
	2nd person singuar
	(all genders)                   kee (kitaaba kee)                     kee (kun  kitaabakeeti)
	
	2nd person plural
	(all genders)                  keessan (kitaaba keessan)              keessan (kun kitaaba keessanidha)
	
	3rd person singular
	masculine
	(all dialects                 isaa (ktaabasaa)                        isaa (kun kitaabasaati)
	
	3rd person singular
	feminine
	(all dialects)             ishee (kitaabashee)                       ishee (kun kitaabashee)
	
	3rd person plural
	all genders               isaanii (kitaabasaanii)                    isaanii (kun kitaabasaaniiti)
	
	3rd person singular
	neuter                    ---------                                 ----------------
	
	\section{Proposed Solution to Genitive of Genitive Problems in Afaan Oromoo}
	
	
	\section{Conclusion}
	
	\newpage
	\bibliographystyle{apacite}
	\bibliography{gen.bib}
	
	\newpage
	\section*{Acronyms and symbols}
	c=palatal,\\
	DEF=definite\\
	F=feminine
	lit.=literal\\
	M=masculine
	plosive, affricate\\
	GEN=genitive\\
	MID=middle\\
	pl=plural\\
	sing=singular\\
	
\end{document}
